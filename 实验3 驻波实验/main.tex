\documentclass[12pt]{article}

\usepackage[a4paper]{geometry}
\geometry{left=2.0cm,right=2.0cm,top=2.5cm,bottom=2.5cm}

\usepackage{ctex}
\usepackage{amsmath,amsfonts,graphicx,subfigure,amssymb,bm,amsthm}
\usepackage{algorithm,algorithmicx}
\usepackage[noend]{algpseudocode}
\usepackage{fancyhdr}
\usepackage{mathrsfs}
\usepackage{mathtools}
\usepackage[framemethod=TikZ]{mdframed}
\usepackage{fontspec}
\usepackage{adjustbox}
\usepackage{breqn}
\usepackage{fontsize}
\usepackage{tikz,xcolor}
\usepackage{hyperref}
\hypersetup{hidelinks}
\usepackage{listings}
\usepackage{textcomp}
\usepackage{siunitx}
\usepackage{float}
\usepackage{booktabs}
\usepackage{multirow}

\definecolor{dkgreen}{rgb}{0,0.6,0}
\definecolor{gray}{rgb}{0.5,0.5,0.5}
\definecolor{mauve}{rgb}{0.58,0,0.82}

\lstset{frame=tb,
    language=Python,
    aboveskip=3mm,
    belowskip=3mm,
    showstringspaces=false,
    columns=flexible,
    basicstyle={\small\ttfamily},
    numbers=left,
    numberstyle=\tiny\color{gray},
    keywordstyle=\color{blue},
    commentstyle=\color{dkgreen},
    stringstyle=\color{mauve},
    breaklines=true,
    breakatwhitespace=true,
    tabsize=4,
    breaklines
}

\setmainfont{Palatino Linotype}
\setCJKmainfont{SimHei}
\setCJKsansfont{Songti}
\setCJKmonofont{SimSun}
\punctstyle{kaiming}

\renewcommand{\emph}[1]{\begin{kaishu}#1\end{kaishu}}

%改这里可以修改实验报告表头的信息
\newcommand{\experiName}{驻波实验}
\newcommand{\supervisor}{吴天涯}
\newcommand{\name}{王致力}
\newcommand{\studentNum}{2021K8009908004}
\newcommand{\class}{3}
\newcommand{\group}{03}
\newcommand{\seat}{09}
\newcommand{\dateYear}{2023}
\newcommand{\dateMonth}{2}
\newcommand{\dateDay}{25}
\newcommand{\room}{721}
\newcommand{\others}{$\square$}
%% 如果是调课、补课, 改为: $\square$\hspace{-1em}$\surd$
%% 否则, 请用: $\square$
%%%%%%%%%%%%%%%%%%%%%%%%%%%

\begin{document}

%若需在页眉部分加入内容, 可以在这里输入
% \pagestyle{fancy}
% \lhead{\kaishu 测试}
% \chead{}
% \rhead{}
\begin{center}
\LARGE \bf 《\, 基\, 础\, 物\, 理\, 实\, 验\, 》\, 实\, 验\, 报\, 告
\end{center}

\begin{center}
    \noindent \emph{实验名称}\underline{\makebox[25em][c]{\experiName}}
    \emph{指导教师}\underline{\makebox[8em][c]{\supervisor}}\\
    \emph{姓名}\underline{\makebox[6em][c]{\name}}%%如果名字比较长, 可以修改box的长度"5em"
    \emph{学号}\underline{\makebox[10em][c]{\studentNum}}
    \emph{分班分组及座号} \underline{\makebox[5em][c]{\class \ -\ \group \ -\ \seat }\emph{号}} (\emph{例}:\, 1\,-\,04\,-\,5\emph{号})\\
    \emph{实验日期} \underline{\makebox[3em][c]{\dateYear}}\emph{年}
    \underline{\makebox[2em][c]{\dateMonth}}\emph{月}
    \underline{\makebox[2em][c]{\dateDay}}\emph{日}
    \emph{实验地点}\underline{{\makebox[4em][c]\room}}
    \emph{调课/补课} \underline{\makebox[3em][c]{\others\ 是}}
    \emph{成绩评定} \underline{\hspace{5em}}
    {\noindent}
    \rule[8pt]{17cm}{0.2em}
\end{center}

\begin{center}
{\Large \textbf{第一部分 \quad 弦上驻波实验}}
\end{center}

\section{实验目的}
\begin{enumerate}
    \item 观察在两端固定的弦线上形成的驻波现象,了解弦线达到共振和形成稳定驻波的条件;
    \item 测定弦线上横波的传播速度;
    \item 用实验的方法确定弦线作受迫振动时共振频率与半波长个数$n$、弦线有效长度、张力及弦密度之间的关系;
    \item 用对数作图和最小乘法对共振频率与张力关系的实验结果作线性拟合,处理数据,并给出结论。
\end{enumerate}

\section{实验器材}

本实验实验装置由弦音计、信号发生器、双踪示波器组成,此外还有天平等测量工具。

\begin{enumerate}
    \item 弦音计由吉他弦、固定吉他弦的支架和基座、琴码、砝码支架、驱动线圈、探测线圈和砝码等组成。其中驱动线圈通过信号发生器提供一定频率的功率信号产生交变磁力,使得金属弦线振动;探测线圈将弦线的振动转换为电信号,通过示波器进行观察。
    \begin{figure}[htbp]
        \centering
        \includegraphics[width=0.5\textwidth]{1-2-1.png}
        \caption{弦线所受张力的示意图}
    \end{figure}
    \item 实验室使用的仪器为低频功率信号发生器,其输出信号的频率从 $10\,\mathrm{Hz}$ 到 $1\,\mathrm{KHz}$。本仪器用来为驱动线圈提供上述频率范围中具有一定功率的正弦信号。
    \item 双踪示波器用于观察信号源的波形,并显示由探测线圈接收到的弦线振动的波形,进而可以及时观察弦线的振动现象。
\end{enumerate}

\section{实验原理}

将一弦线两端固定,以一定的张力绷紧。在一端附近使得弦线作振幅恒定的连续简谐振动,将有连续的横波波列从该端向另一端传播。前进波传播到另一端时即会发生反射,回到原端点时再次反射,如此不断重复下去。弦线上既有前进波,又有无数的反射波。如果弦线的长度与波长之间满足某种关系,使得前进波与许多反射波都具有相同的相位时,弦线上各点作振幅各自恒定的简谐振动。

这种情况称为驻波现象,弦线上振幅最大的点称为波腹,振幅为零的点称为波节。相邻两波节(或波腹)的间隔距离$D$为波长$\lambda$的一半,称为半波长,即$\lambda=2D$。

由于弦线两端固定,故而弦线两端均为波节,故弦线长度应为半波长的整数倍,记弦线长度为$ L $,则

\begin{equation}\label{eq:1}
    \lambda=\frac{2L}{n},\quad n=1,2,3,\cdots
\end{equation}

若振动频率为$f$,则横波沿弦线传播的速度为$v=f\lambda$。根据波动理论,假设拉紧的弦上张力为$ T $,弦线的线密度为$ \mu $,则沿弦线传播的横波应满足下述运动方程:

\begin{equation}\label{eq:2}
    \begin{aligned}
        \frac{\partial^2y}{\partial t^2}&=\frac{T\partial^2y}{\mu\partial x^2}\\
        \frac{\partial^2y}{\partial t^2}&=v^2\frac{\partial^2y}{\partial x^2}
    \end{aligned}
\end{equation}

其中$x$为波在传播方向(与弦线平行)的位置坐标,$y$为振动引起的位移。比较公式(\ref{eq:1})与(\ref{eq:2}),可以得到波的传播速度:

\begin{equation}\label{eq:3}
    v=\sqrt{\frac{T}{\mu}}
\end{equation}

又因为$v=f\lambda$,可以得到波长与张力及线密度之间的关系:

\begin{equation}\label{eq:4}
    \lambda=\frac{1}{f}\sqrt{T}{\mu}
\end{equation}

两边同时取对数得:

\begin{equation}\label{eq:5}
    \ln\lambda=\frac{1}{2}\ln T - \frac{1}{2}\ln\mu-\ln f
\end{equation}

该公式可以用于实验验证$\lambda$与$f$的关系。

\section{实验内容}

\begin{enumerate}
    \item 认识和调节仪器。
    \item 测定所用弦线的线密度。选取与实验装置中所用弦线材料相同的弦线,保证直径相同,且只取吉他弦中段约$70-80\,\mathrm{cm}$的专用样品。记测得的弦线质量为$m$,弦线长$L$,则线密度为$\mu=\frac{m}{L}$。
    \item 观察弦线上的驻波。固定弦上张力$ T $与波的有效长度$ L $,调节信号发生器的输出频率,观察在两端固定的弦线上形成了$n\,(n=1,2,3,\cdots) $个波腹的稳定驻波。
    \item 测定弦线上横波的传播速度。有两种方法测定传播速度$v$:
    
    第一种方法是将张力$T$及所测线密度$\mu$带入$v=\frac{T}{\mu}$求解。

    第二种方法是先测出共振频率$f$再测$L$,并用(\ref{eq:1})推出$\lambda$,然后带入式(\ref{eq:2})求出$v$。
    \item 确定弦线作受迫振动时的共振频率(只取基频,即$n=1$)与张力之间的关系(此时固定弦线张力和弦线密度),并记录数据。
    \item 确定弦线作受迫振动时的共振频率(只取基频,即$n=1$)与弦线有效长度之间的关系(此时固定弦线张力和弦线密度),并记录数据。
    \item 确定弦线作受迫振动时的共振频率与半波长个数$n$之间的关系(此时固定弦线张力、弦线长度、弦线密度),并记录数据。
\end{enumerate}

\section{实验结果与数据处理}
\subsection{线密度测试}
\begin{table}[htbp]
    \centering
    \begin{tabular}{|c|c|c|c|c|}
        \hline
        弦号 & 质量(g) & 长度(mm) & 直径(mm) & 线密度(Kg/m) \\
        \hline
        8 & 0.305 & 86.4 & 0.805 & $3.53\times10^{-3}$ \\
        \hline        
    \end{tabular}
    \caption{线密度测试}
\end{table}

\subsection{波速的测量}
将琴码放在150mm和650mm的地方,将砝码放在第2-4格,测基频$f_1$,倍频$f_2$,$f_3$,计算波速的实验值($v=\lambda f$);根据$v=\sqrt{T/\mu}$,$T=\frac{1}{2}nmg$计算波速的理论值。

实验中观察到的波节如图\ref{fig:2}所示,可以看到弦两侧因为振动较为模糊,而中间较清晰,因此可以认为中间是静止的,这也符合驻波的特点。

\begin{figure}[htbp]
    \centering
    \includegraphics[width=0.8\textwidth]{1-5-1.jpg}
    \caption{实验中观察到的波节,在弦的中点处}
    \label{fig:2}
\end{figure}

首先我们用天平测得砝码的质量为508.40g。然后我们分别用上述两个公式计算波速,其中对于公式$v=\lambda f$,我们测定$n=1,2,3$时不同的频率,计算对应的波速$v$并取平均值,实验结果如表\ref{tab:2}所示。对比两种公式的计算结果可知,在误差允许范围内,通过两种方法计算出的波速基本相等。

\newpage

\begin{table}[htbp]
    \centering
    \begin{tabular}{|c|c|c|c|c|c|c|}
        \hline
        砝码位置 & $f_1$(Hz) & $f_2$(Hz) & $f_3$(Hz) & 波速$v=\lambda f$ & 张力$T$(N) & 波速$v=\sqrt{T/\mu}$ \\
        \hline
        2    & 37.8 & 76.2 & 114.8 & 38.06 & 4.98 & 37.57 \\
        \hline
        3    & 47.0 & 93.9 & 141.2 & 47.01 & 7.47 & 46.01 \\
        \hline
        4    & 55.5 & 109.8 & 165.2 & 55.16 & 9.96 & 53.13 \\
        \hline
    \end{tabular}%
    \label{tab:2}
    \caption{波速的测试}
\end{table}%

\subsection{频率和有效长度的关系}
在上述实验中,法码放在第2格,改变有效长度,测试频率$f_1$的变化。

\begin{table}[htbp]
    \centering
    \begin{tabular}{|c|c|c|c|c|c|}
        \hline
        $L$ & 640mm & 480mm & 320mm & 260mm & 160mm \\
        \hline
        $f_1$ & 29.4Hz & 41.2Hz & 59.4Hz & 81.8Hz & 123.8Hz \\
        \hline
    \end{tabular}
    \caption{频率和有效长度的关系}
\end{table}

可以看到$f_1$与$L$负相关。根据$v=\lambda f_1 = \sqrt{T/\mu}$可知,$\ln f_1 = -\ln L + \frac{1}{2} \ln T - \frac{1}{2} \ln \mu$,而在张力$T$恒定的情况下,理论上$\ln f_1$与$\ln L$呈线性关系且斜率为-1。

用Python中的SciPy库进行线性拟合并用Python的Matplotlib库画图,结果如图\ref{fig:3}所示。数据点基本近似都落在直线上,且拟合出的直线斜率为-1.03,与-1近似,故在误差允许范围内可以认为$\ln f_1$与$\ln L$呈线性关系且斜率为-1,即$f_1 \propto 1/L$。

\begin{figure}[htbp]
    \centering
    \includegraphics[width=0.5\textwidth]{1-5-2.png}
    \caption{$\ln L$与$\ln f_1$的线性拟合结果}
    \label{fig:3}
\end{figure}

\subsection{频率和张力的关系}
固定有效长度L=400mm,将琴码放在200mm和600mm的地方,然后将砝码放在1-5格式,测频率$f_1$。

\begin{table}[h!]
    \centering
    \begin{tabular}{|c|c|c|c|c|c|}
        \hline
        位置   & 1    & 2    & 3    & 4    & 5 \\
        \hline
        $T$(N)    & 2.49  & 4.98  & 7.47  & 9.96  & 12.46  \\
        \hline
        $f_1$(Hz)   & 33.6 & 49.2 & 60.3 & 70.1 & 71.2 \\
        \hline
    \end{tabular}%
    \caption{频率和张力的关系}
\end{table}%

注意到$T$与$f_1$正相关。根据公式\ref{eq:5},$\ln f_1=\frac{1}{2}\ln T - \frac{1}{2}\ln\mu - \ln\lambda$。在固定有效长度后,$\lambda$和$\mu$恒定,故$\ln f_1$与$\ln T$成线性关系,且斜率为$\frac{1}{2}$。

用Python中的SciPy库进行线性拟合并用Python的Matplotlib库画图,结果如图\ref{fig:4}所示。可见除了最后一个点误差较大外,整体的线性关系较好,且斜率为0.49与0.5近似,故在误差允许范围内可以认为$\ln f_1$与$\ln T$呈线性关系且斜率为0.5,即$f_1 \propto \sqrt{T}$。

\begin{figure}[htbp]
    \centering
    \includegraphics[width=0.5\textwidth]{1-5-3.png}
    \caption{$\ln T$与$\ln f_1$的拟合结果}
    \label{fig:4}
\end{figure}

\subsection{频率与线密度的关系}
固定有效长度L=400mm,将琴码放在200mm和600mm的地方,将砝码放在第2格,从5位同学的实验数据中获取不同粗细琴弦的$f_2$数据。比起实验记录表,这里将5组数据按照线密度从小到大排列,以便观察$\mu$与$f_2$之间的关系。

\begin{table}[htbp]
    \centering
    \begin{tabular}{|c|c|c|c|c|c|}
        \hline
        弦号   & 11   & 12   & 8    & 5    & 7 \\
        \hline
        直径(mm) & 0.593 & 0.851 & 0.805 & 0.605 & 1.07 \\
        \hline
        $\mu$(Kg/m) & 0.001875 & 0.00189 & 0.00353 & 0.00559 & 0.00582 \\
        \hline
        $f_2$(Hz)   & 64.263 & 59.952 & 49.208 & 39.166 & 37.651 \\
        \hline
    \end{tabular}%
    \caption{频率和线密度的关系}
\end{table}

注意到$T$与$f_1$负相关。根据公式\ref{eq:5},$\ln f_2 = -\frac{1}{2}\ln\mu + \frac{1}{2}\ln T - \ln\lambda$。在固定有效长度和砝码位置后,$\lambda$和$T$恒定,故$\ln f_2$与$\ln \mu$成线性关系,且斜率为$-\frac{1}{2}$。

用Python中的SciPy库进行线性拟合并用Python的Matplotlib库画图,结果如图\ref{fig:5}所示。在5个数据中,弦号为12的数据因为和其他数据偏差太大而被删除。删除这一数据点后,整体的线性关系较好,且斜率为-0.47与-0.5近似,故在误差允许范围内可以认为$\ln f_2$与$\ln T$呈线性关系且斜率为-0.5,即$f_2 \propto \sqrt{1/\mu}$。

\begin{figure}[htbp]
    \centering
    \includegraphics[width=0.5\textwidth]{1-5-4.png}
    \caption{$\ln \mu$与$\ln f_2$的拟合结果}
    \label{fig:5}
\end{figure}

\section{思考题}
\begin{enumerate}   
    \item {\kaishu 调节振动源上的振动频率和振幅大小后对弦线振动会产生什么影响?}
    
    弦线上各点的振动由振动源的振动引起。若调节振动源的频率,则会影响弦线上各点处前进波与反射波的叠加情况,一定条件下,驻波现象产生;调节振动源的振幅,将影响弦线上各点的振动振幅,但对有无驻波现象没有影响。
    
    \item {\kaishu 如何来确定弦线上的波节点位置?}
    
    观察弦线震动情况,即振幅几乎为0、基本静止的点即为波节点。

    \item {\kaishu 在弦线上出现驻波的条件是什么?在实验中为什么要把弦线的振动调到驻波现在最稳定、最显著的状态?}
    
    根据实验原理部分可知出现驻波的条件为$ \lambda = 2L/n\;(n=1,2,3,\cdots) $,即有效长度是半波长的正整数倍。将弦线振动调节至驻波最稳定、最显著的状态是因为这是最接近驻波出现条件的状态,确保各点处前进波和反射波有稳定相位差并形成驻波,使得测得的基频更为准确。

    \item {\kaishu 在弹奏弦线乐器时,发出声音的音调与弦线的长度、粗细、松紧程度由什么关系?为什么?}
    
    在弹奏弦线乐器时,弦线长度越短,驻波波长越小,频率越大;弦的直径越小,弦线的线密度越小,驻波频率越大;琴弦越紧,弦线上拉力越大,驻波频率越大。发出声音的音调越高。而频率越大,弦线乐器的音调越高。综上,乐器的音调与$ f $与弦线长度成负相关、弦线直径成负相关、弦线松紧程度成正相关。
    
    \item {\kaishu 若样品弦线与装置上的弦线直径略有差别,请判断是否需要修正,如何进行?}
    
    如果保证样品弦线与装置上的弦线密度相等,则弦的直径会影响单位长度上弦线的体积,从而影响线密度,因此需要修正。
    
    设一段弦线的密度为$\rho$,长度为$l$,直径为$d$,则其体积为$V=\frac{\pi}{4}d^2l$,质量为$m=\frac{\pi}{4}\rho d^2l$,线密度为$\mu=\frac{\pi}{4}\rho d^2$。我们设样品弦线的直径为$d_0$(可以用螺旋测微器直接测量),线密度为$\mu_0$;装置上的弦线直径为$d$,线密度为$\mu$,则:

    \[
        \frac{\mu}{\mu_0}=\frac{\frac{\pi}{4}\rho d^2}{\frac{\pi}{4}\rho d_0^2}=(\frac{d}{d_0})^2
    \]

    因此我们对线密度做出修正$\mu=\mu_0(\frac{d}{d_0})^2$。

    \item {\kaishu 对于某一共振频率,增大或减小频率的调节过程中,振幅最大的频率位置往往不同,如何解释这一现象?}
    
    驻波现象在一定频率范围内都能被观测到。同时信号发生器的输出频率变化可能有延迟,实验过程中来回调节频率也会导致回调误差,导致振幅最大的频率位置不同。
\end{enumerate}

\setcounter{section}{0}
\newpage
\begin{center}
{\Large \textbf{第二部分 \quad 测定介质中的声速}}
\end{center}

\section{实验目的}
\begin{enumerate}
    \item 利用驻波法测定波长;
    \item 利用相位法测定波长;
    \item 计算超声波在空气和水中的传播速率。
\end{enumerate}

\section{实验器材}
SW-2型声速测量仪,信号发生器,示波器。

\section{实验原理}
\subsection{利用驻波法测声速}
将信号发生器输出的正弦电压信号街道超声发射换能器上,经超声发射换能器电声转换为超声波并发射出去。接收换能器通过声电转换将声波信号变为为电压信号,传入示波器。
	
由声波传输理论可知,从发射换能器发出一定频率的平面声波,经过介质传播到达接收换能器。如果接收面与发生面严格平行,入射波在接收面上垂直反射,入射波、反射波相互干涉形成驻波,此时两换能器之间距离恰好等于其声波半波长的整数倍。在声驻波中,波腹处声压最小,波节处生涯最大,因此可通过接收换能器端面声压的变化来判断超声波是否形成驻波。

转动鼓轮,改变两只换能器间的距离,在一系列特定的距离上,将会出现稳定驻波,记录下出现最大电压数值时标尺上的刻度,相邻两次最大值对应的刻度值之差即为半波长。

根据公式$v=\lambda f$,频率$f$已知,根据上述方法可以求出波长$\lambda$,就可以算出超声波的传播速度$v$。

\subsection{利用相位法测声速}
将发射波和接收波同时输入示波器,以X-Y模式显示。两波的频率相同,相位不同。当接受点与发射点的距离变化恰等于一个波长时,相位差正好是$2\pi$。
	
实验时,通过改变发射器和接收器之间的距离,观察李萨如图形的变化进而观察相位变化,当相位改变$ \pi $时,相应距离的改变量即为半波长。根据公式$ v=\lambda f $即可可求出波速。

相位变化时,部分李萨如图形如下:

\begin{figure}[htbp]
    \centering
    \includegraphics[width=0.8\textwidth]{2-2-1.png}
    \caption{频率相同、相位不同时的李萨如图形}
\end{figure}

\subsection{声速的理论值}
利用声速在空气中的理论公式可以计算空气中声速的理论值:
\begin{equation}\label{eq:6}
    v=v_0\sqrt{\frac{T}{T_0}}=v_0\sqrt{1+\frac{t}{273.15}}
\end{equation}

\section{实验内容}
\begin{enumerate}
    \item 利用驻波法测超声波在空气中的波速;
    \item 利用相位法测超声波在空气中的波速;
    \item 利用驻波法或相位法测超声波在水中的波速;
    \item 利用逐差法处理实验数据。
\end{enumerate}

\section{实验结果与数据处理}
\subsection{测超声波在空气中的波速}
该实验中相邻两个$L_i$数据点对应驻波法中超声波在示波器某处从波峰变为波谷的过程,以及位相法中李萨如图形从$\Delta\phi=0$到$\Delta\phi=\pi$的过程,两者的差对应半波长。因此利用逐差法比较$L_{i+5}$与$L_i$,两者的差对应2.5个波长。将逐差法得到的波长取平均值得到$\lambda$,即:

\[
    \lambda=\frac{\sum_{i=1}^{5}2(L_{i+5}-L_i)/5}{5}
\]

最后用$v=\lambda f$求出波速并与理论值相比较。实验数据与计算结果如表\ref{tab:air}所示。其中驻波法的测量结果与理论值的相对误差为1.96\%,位相法的测量结果与理论值的相对误差为1.50\%,在误差允许范围内可以接受。

\begin{table}[htbp]
    \centering
    \begin{tabular}{|ccccc|}
    \hline
    \multicolumn{5}{|c|}{$f=40$kHz,\quad 室温$t$=259$^\circ$ C,\quad $V_{\text{理论值}}=346.81$m/s}                                                                                    \\ \hline
    \multicolumn{1}{|c|}{i}    & \multicolumn{1}{c|}{驻波法$L_i$(mm)} & \multicolumn{1}{c|}{$\lambda_i$(mm)}           & \multicolumn{1}{c|}{位相法$L_i$(mm)} & $\lambda_i$(mm)      \\ \hline
    \multicolumn{1}{|c|}{1}    & \multicolumn{1}{c|}{47.100}    & \multicolumn{1}{c|}{8.68}                  & \multicolumn{1}{c|}{48.250}    & 8.90                  \\ \hline
    \multicolumn{1}{|c|}{2}    & \multicolumn{1}{c|}{51.300}    & \multicolumn{1}{c|}{8.80}                  & \multicolumn{1}{c|}{52.900}    & 8.80                  \\ \hline
    \multicolumn{1}{|c|}{3}    & \multicolumn{1}{c|}{55.350}    & \multicolumn{1}{c|}{8.94}                  & \multicolumn{1}{c|}{57.400}    & 8.72                  \\ \hline
    \multicolumn{1}{|c|}{4}    & \multicolumn{1}{c|}{59.550}    & \multicolumn{1}{c|}{8.98}                  & \multicolumn{1}{c|}{61.500}    & 8.72                  \\ \hline
    \multicolumn{1}{|c|}{5}    & \multicolumn{1}{c|}{64.300}    & \multicolumn{1}{c|}{8.80}                  & \multicolumn{1}{c|}{65.900}    & 8.88                  \\ \hline
    \multicolumn{1}{|c|}{6}    & \multicolumn{1}{c|}{68.800}    & \multicolumn{1}{c|}{\multirow{5}{*}{\shortstack{平均值\\8.84}}} & \multicolumn{1}{c|}{70.500}    & \multirow{5}{*}{\shortstack{平均值\\8.80}} \\ \cline{1-2} \cline{4-4}
    \multicolumn{1}{|c|}{7}    & \multicolumn{1}{c|}{73.300}    & \multicolumn{1}{c|}{}                      & \multicolumn{1}{c|}{74.950}    &                       \\ \cline{1-2} \cline{4-4}
    \multicolumn{1}{|c|}{8}    & \multicolumn{1}{c|}{77.700}    & \multicolumn{1}{c|}{}                      & \multicolumn{1}{c|}{79.200}    &                       \\ \cline{1-2} \cline{4-4}
    \multicolumn{1}{|c|}{9}    & \multicolumn{1}{c|}{82.000}    & \multicolumn{1}{c|}{}                      & \multicolumn{1}{c|}{83.300}    &                       \\ \cline{1-2} \cline{4-4}
    \multicolumn{1}{|c|}{10}   & \multicolumn{1}{c|}{86.300}    & \multicolumn{1}{c|}{}                      & \multicolumn{1}{c|}{88.100}    &                       \\ \hline
    \multicolumn{1}{|c|}{测量结果} & \multicolumn{2}{c|}{$v$=353.60m/s}                                            & \multicolumn{2}{c|}{$v$=352.00m/s}                       \\ \hline
    \end{tabular}
    \label{tab:air}
    \caption{空气中超声波的测试}
\end{table}

\subsection{测超声波在水中的速度}
在搭建实验装置时,需要注意将两个探头固定好,使之不能绕着顶部做类似单摆的旋转运动。否则每次转动鼓轮时,探头的晃动都会改变两探头的实际距离,对实验结果造成误差。

由于李萨如图形相对更好观察,本实验使用了位相法以减小偶然误差。该实验中相邻两个$L_i$数据点对应李萨如图形从$\Delta\phi=0$到$\Delta\phi=\pi$的过程,两者的差对应半波长。类似于空气中波速的测量,用逐差法处理实验数据,实验数据与计算结果如表\ref{tab:water}所示。

% Please add the following required packages to your document preamble:
% \usepackage{multirow}
\begin{table}[htbp]
    \centering
    \begin{tabular}{|ccc|}
    \hline
    \multicolumn{3}{|c|}{方法:位相法,\quad $f$=1.7MHz,\quad 室温$t$=25.9$^\circ$C}                         \\ \hline
    \multicolumn{1}{|c|}{i}  & \multicolumn{1}{c|}{刻度值$L_i$(mm)} & $\lambda_i$(mm)        \\ \hline
    \multicolumn{1}{|c|}{1}  & \multicolumn{1}{c|}{71.140}    & 0.944                   \\ \hline
    \multicolumn{1}{|c|}{2}  & \multicolumn{1}{c|}{72.720}    & 0.908                   \\ \hline
    \multicolumn{1}{|c|}{3}  & \multicolumn{1}{c|}{72.120}    & 0.920                   \\ \hline
    \multicolumn{1}{|c|}{4}  & \multicolumn{1}{c|}{72.570}    & 0.936                   \\ \hline
    \multicolumn{1}{|c|}{5}  & \multicolumn{1}{c|}{73.050}    & 0.908                   \\ \hline
    \multicolumn{1}{|c|}{6}  & \multicolumn{1}{c|}{73.500}    & \multirow{5}{*}{\shortstack{平均值\\0.9232}} \\ \cline{1-2}
    \multicolumn{1}{|c|}{7}  & \multicolumn{1}{c|}{73.980}    &                         \\ \cline{1-2}
    \multicolumn{1}{|c|}{8}  & \multicolumn{1}{c|}{74.420}    &                         \\ \cline{1-2}
    \multicolumn{1}{|c|}{9}  & \multicolumn{1}{c|}{74.910}    &                         \\ \cline{1-2}
    \multicolumn{1}{|c|}{10} & \multicolumn{1}{c|}{75.320}    &                         \\ \hline
    \multicolumn{3}{|c|}{测量结果:$V_{\text{实验值}}$=1569.44m/s}                              \\ \hline
    \end{tabular}
    \label{tab:water}
    \caption{水中超声波波速的测试}
\end{table}

\setcounter{section}{0}
\begin{center}
{\Large \textbf{第三部分 \quad 实验总结}}
\end{center}

%\section{实验总结}
本次实验除了仪器本身存在的一些问题之外,个人在操作中最大的失误就是在两个实验中进行了回调,导致正向调节和反向调节得到的结果不一致需要重做。尤其是在声速测量实验中,反向调节耽误了我较多的时间,甚至导致最后一个实验直接重做。我之前并不了解回调误差的概念,讲义上也没有明确,但事后和物理系的同学交流后,认识到在物理实验中一般来说都应该尽可能避免将试验仪器回调,以免造成不必要的误差。同时我也建议在讲义中的注意事项部分强调一下这个问题,以免实验经验不足的同学踩坑。

而仪器对本实验的影响也比较大,尤其是在利用驻波法测声速的实验中,超声波发射器和接收器并不精确,导致示波器读取到的信号不稳定,难以度数而且容易造成较大的误差。此外我也发现周围的同学在弦上驻波实验中选到的装置弦太粗导致实验效果不明显,或许可以考虑更新部分实验设备。

最后,本篇实验报告全部用\LaTeX{}排版,用Python进行数据处理和绘图。这里感谢21级计算机专业的吉骏雄同学和21级人工智能专业的林诚皓同学提供实验报告表头所用模板,也感谢16级物理系樊兆兴学长对于部分\LaTeX{}相关问题的解答!

\end{document}