\documentclass[12pt]{article}

\usepackage[a4paper]{geometry}
\geometry{left=2.0cm,right=2.0cm,top=2.5cm,bottom=2.5cm}

\usepackage{ctex}
\usepackage{amsmath,amsfonts,graphicx,subfigure,amssymb,bm,amsthm}
\usepackage{algorithm,algorithmicx}
\usepackage[noend]{algpseudocode}
\usepackage{fancyhdr}
\usepackage{mathrsfs}
\usepackage{mathtools}
\usepackage[framemethod=TikZ]{mdframed}
\usepackage{fontspec}
\usepackage{adjustbox}
\usepackage{breqn}
\usepackage{fontsize}
\usepackage{tikz,xcolor}
\usepackage{hyperref}
\hypersetup{hidelinks}

\setmainfont{Palatino Linotype}
\setCJKmainfont{SimHei}
\setCJKsansfont{Songti}
\setCJKmonofont{SimSun}
\punctstyle{kaiming}

\renewcommand{\emph}[1]{\begin{kaishu}#1\end{kaishu}}

%改这里可以修改实验报告表头的信息
\newcommand{\experiName}{RLC电路的谐振与暂态过程}
\newcommand{\supervisor}{李国强}
\newcommand{\name}{王致力}
\newcommand{\studentNum}{2021K8009908004}
\newcommand{\class}{3}
\newcommand{\group}{03}
\newcommand{\seat}{10}
\newcommand{\dateYear}{2022}
\newcommand{\dateMonth}{11}
\newcommand{\dateDay}{2}
\newcommand{\room}{709}
\newcommand{\others}{$\square$}
%% 如果是调课、补课, 改为: $\square$\hspace{-1em}$\surd$
%% 否则, 请用: $\square$
%%%%%%%%%%%%%%%%%%%%%%%%%%%

\begin{document}

%若需在页眉部分加入内容, 可以在这里输入
% \pagestyle{fancy}
% \lhead{\kaishu 测试}
% \chead{}
% \rhead{}

\begin{center}
    \LARGE \bf 《\, 基\, 础\, 物\, 理\, 实\, 验\, 》\, 实\, 验\, 报\, 告
\end{center}

\begin{center}
    \noindent \emph{实验名称}\underline{\makebox[25em][c]{\experiName}}
    \emph{指导教师}\underline{\makebox[8em][c]{\supervisor}}\\
    \emph{姓名}\underline{\makebox[6em][c]{\name}}%%如果名字比较长, 可以修改box的长度"5em"
    \emph{学号}\underline{\makebox[10em][c]{\studentNum}}
    \emph{分班分组及座号} \underline{\makebox[5em][c]{\class \ -\ \group \ -\ \seat }\emph{号}} (\emph{例}:\, 1\,-\,04\,-\,5\emph{号})\\
    \emph{实验日期} \underline{\makebox[3em][c]{\dateYear}}\emph{年}
    \underline{\makebox[2em][c]{\dateMonth}}\emph{月}
    \underline{\makebox[2em][c]{\dateDay}}\emph{日}
    \emph{实验地点}\underline{{\makebox[4em][c]\room}}
    \emph{调课/补课} \underline{\makebox[3em][c]{\others\ 是}}
    \emph{成绩评定} \underline{\hspace{5em}}
    {\noindent}
    \rule[8pt]{17cm}{0.2em}
\end{center}

\section{实验目的}
\begin{itemize}
    \item 研究RLC电路的谐振现象。
    \item 了解RLC电路的相频特性与幅频特性。
    \item 用数字存储示波器观察RLC串联电路的暂态过程,理解阻尼振动规律。
\end{itemize}

\section{实验器材}
标准电感、标准电容,$ 100\,\Omega $标准电阻,电阻箱,电感箱,函数发生器,示波器,数字多用表,导线等。

\section{实验原理}
\subsection{串联谐振}
RLC串联电路如图\ref{fig:1}所示。根据电磁学课程中介绍的交流电复数解法,其总阻抗$|Z|$、电压$u$和电流$i$之间相位差$\varphi$、电流$i$分别为:

\[
    \begin{aligned}
        |Z|&=\sqrt{R^2+(\omega L-1/\omega C)^2} \\
        \varphi&=\arctan\frac{\omega L-1/\omega C}{R} \\
        i&=\frac{u}{\sqrt{R^2+(\omega L-1/\omega C)^2}}
    \end{aligned}
\]

\begin{figure}[htbp]\label{fig:1}
    \centering
    \includegraphics[height=150pt]{3-1.png}
    \caption{RLC串联电路图}
\end{figure}

其中$\omega=2\pi f$为角频率,$|Z|,\,\varphi,\,i$都是$f$的函数。当电路中其他元件参量都取确定值的情况下,它们的特性完全取决于频率。

$\varphi,\,i$关于$f=\frac{\omega}{2\pi}$的函数分别称为相频特性曲线、幅频特性曲线。我们对$|Z|$求关于自变量$\omega$的函数,求得$|Z|$的极值点$\omega_0$:

\[
    \begin{aligned}
        \frac{\mathrm{d}|Z|}{\mathrm{d}\omega}&=\frac{2(\omega L-1/\omega C)(L+1/\omega^2C)}{2\sqrt{R^2+(\omega L-1/\omega C)^2}} \\
        \left.\frac{\mathrm{d}|Z|}{\mathrm{d}\omega}\right|_{\omega = \omega_0}&=0 \\
        \omega_0&=\frac{1}{\sqrt{LC}} \\
        f_0&=\frac{1}{2\pi}\omega_0=\frac{1}{2\pi\sqrt{LC}}
    \end{aligned}
\]

此时电路的总阻抗$|Z|$最大$=R$,电流$i$最小$=\frac{u}{R}$,相位差$\varphi=0$。即电流与电压的相位完全相同,整个电路呈纯电阻性。我们称此时的电路状态为串联谐振。

在串联状态下,电路的电压关系如下:
\[
    \begin{aligned}
        &u_L=i_m|Z_L|=\frac{\omega_0L}{R}u,&\frac{u_L}{u}=\frac{\omega_0L}{R}=\frac{1}{R}\sqrt{\frac{L}{C}} \\
        &u_C=i_m|Z_C|=\frac{1}{R\omega_0C}u,&\frac{u_C}{u}=\frac{1}{R\omega_0C}=\frac{1}{R}\sqrt{\frac{L}{C}}
    \end{aligned}
\]

定义谐振电路的品质因数$Q$:

\[
    Q=\frac{u_L}{u}=\frac{u_C}{u}=\frac{\omega_0L}{R}=\frac{1}{R\omega_0C}
\]

$Q$值标志谐振电路的储耗能特性、电压分配特性、频率选择性。

\subsection{并联谐振}

如图\ref{fig:2}所示电路,其总阻抗$|Z_p|$、电压$u$与电流$i$之间的相位差$\varphi$、电压$u$(或电流$i$)分别为

\[
    \begin{aligned}
        |Z_p|&=\sqrt{\frac{R^2+(\omega L)^2}{(1-\omega^2LC)^2+(\omega CR)^2}} \\
        \omega &=\arctan\frac{\omega L-\omega C[R^2+(\omega L)^2]}{R} \\
        u &=i|Z_p|=\frac{u_{R'}}{R'}|Z_p|
    \end{aligned}
\]

\begin{figure}[htbp]\label{fig:2}
    \centering
    \includegraphics[height=150pt]{3-2.png}
    \caption{RLC并联电路图}
\end{figure}

它们都是$\omega$的函数。类似串联谐振电路,当$\omega=0$时,电流和电压同相位,整个电路呈纯电阻性,即发生谐振。此时的$\omega$为并联谐振电路的角频率$\omega_p$,$f$为并联谐振电路的频率$f$:

\[
    \begin{aligned}
        \omega_p=2\pi f_p=\sqrt{\frac{1}{LC}-\left(\frac{R}{L}\right)^2}=\omega_0\sqrt{1-\frac{1}{Q^2}}
    \end{aligned}
\]

其中$\omega_0=2\pi f_0=1/\sqrt{LC}$,$Q=\omega L/R=\sqrt{L/C}/R$。可见,并联谐振频率$f_p$与$f_0$稍有不同,当$Q>>1$时,$\omega_p\approx\omega_0,\,f_p\approx f_0$。

\subsection{RLC电路的暂态过程}

考虑串联RLC电路,如图\ref{fig:3}。

\begin{figure}[htbp]\label{fig:3}
    \centering
    \includegraphics[height=120pt]{3-3.png}
    \caption{观察RLC暂态过程所用电路图}
\end{figure}

先观察放电过程,即开关 S 先合向“1”使电容充电至 $E$,然后把 S 倒向“2”,电容就在闭合的 RLC 电路中放电。电路方程为

\[
    \begin{aligned}
        L\frac{\mathrm{d}i}{\mathrm{d}t}+Ri+u_C&=0 \\
        L\frac{\mathrm{d}^2u_C}{\mathrm{d}t^2}+R\frac{\mathrm{d}u_C}{\mathrm{d}t}+u_C&=0
    \end{aligned}
\]

这是一个二阶常微分方程,有初始条件$t=0,\,u_C=E,\,\frac{\mathrm{d}u_C}{\mathrm{d}t}=0$。方程的解分3种情况,对应下图中的三种曲线:

\begin{figure}[htbp]\label{fig:4}
    \centering
    \includegraphics[height=150pt]{3-4.png}
    \caption{RLC暂态过程中的三种阻尼曲线}
\end{figure}

\begin{itemize}
    \item $R^2<4L/C$属于阻尼较小的情况。此时我们定义阻尼系数$\zeta=\frac{R}{2}\sqrt{\frac{C}{L}}<1$,时间常量$\tau=\frac{2L}{R}$,则方程的解为:
    
    \[
        \begin{aligned}
            u_C&=\sqrt{\frac{4L}{4L-R^2C}}Ee^{-\frac{t}{\tau}}cos(\omega t+\varphi) \\
            \omega&=\frac{1}{LC}\sqrt{1-\frac{R^2C}{4L}}
        \end{aligned}
    \]

    该情况对应图\ref{fig:4}中的图线I,称为欠阻尼状态。

    \item $R^2>4L/C$时,阻尼系数$\zeta>1$,方程的解为:
    
    \[
        u_C=\sqrt{1-\frac{R^2C}{4L}}Ee^{-at}\sinh(\beta t+\varphi)
    \]

    其中:

    \[
        \alpha=\frac{R}{2L},\,\beta=\frac{1}{\sqrt{LC}}\sqrt{\frac{R^2C}{4L}-1}
    \]

    该情况对应图\ref{fig:4}中的图线II,称为过阻尼状态。

    \item $R^2=4L/C$,即阻尼系数$\zeta=1$。方程的解为:
    
    \[
        u_C=E(1+\frac{t}{\tau})e^{-\frac{t}{\tau}}
    \]

    该情况对应图\ref{fig:4}中的图线III,称为临界阻尼状态,是以上三种情况中衰减最快的一种。
\end{itemize}

此外,对于充电过程,类似的,我们有如下方程:

\[
    LC\frac{\mathrm{d}^2u_C}{\mathrm{d}t^2}+RC\frac{\mathrm{d}u_C}{\mathrm{d}t}+u_C=E
\]

初始条件为$t=0,\,u_C=E,\,\frac{\mathrm{d}u_C}{\mathrm{d}t}=0$。最终的曲线可以看成放电曲线沿$y=\frac{E}{2}$翻折,这里不再具体写出过程。

\section{实验内容}
本实验中所用的函数发生器、示波器都是接地的(已通过其电源插头与大地连通)。示波器实际上测量的是通道中心线对地的电压,因此利用示波器测量某元件上的电压时,需要留意电路中共地点的位置。一般我们将电阻放在电路的最下游,然后在电流流出电阻的位置同时接上两个示波器输入信号的负极,以达到共地的效果。

\subsection{测RLC串联电路的相频特性和幅频特性曲线}
取$ u_{pp}=2.0\,\mathrm V,\;L=0.1\,\mathrm H,\;C=0.05\,\mu\mathrm F,\;R=100\,\Omega $时,用示波器CH1、CH2通道分别观测$ RLC $串联电路的总电压$ u $和电阻两端电压$ u_R $。注意限制总电压峰值不超过$ 3.0\,\mathrm V $(或有效值不超过0.1\,V),防止串联谐振时产生有危险的高电压。

\begin{enumerate}
    \item 调谐振,改变函数发生器的输出频率,通过CH1与CH2相位差为0,CH2幅度最大来判断谐振与否,记录谐振时的频率$ f_0 $。
    \item 用万用表记录谐振时的电感$ u_L $、电容两端的电压$ u_C $和电源路端电压$ u $,计算$ Q $值。
    \item 保持CH1的幅度为$ 2\,\mathrm V $不变,按照建议的频率点测量CH1与CH2的相位差、CH2的幅度值,并绘制相频曲线和幅频曲线,即$ \varphi-f $图象、$ i-f $图象。
\end{enumerate}

\subsection{测RLC并联电路的相频特性和幅频特性曲线}
取$L=0.1\,\mathrm H,\;C=0.05\,\mu\mathrm F,\;R'=5\,\mathrm k\Omega $。为观测电感与电容并联部分的电压和相位,用CH1测量总电压,用CH2测量$ R' $两端电压,通过示波器面板上的“MATH”将两通道的波形相减,得到并联部分的电压$ u $。
\begin{enumerate}	
	\item 调节函数发生器频率,通过观察CH1$ - $CH2与CH2相位差为0,CH2的幅度最小来判断谐振点,记录此时的频率。
	
	\item 保持CH1测得的总电压(即$u+u'$)为2\,V不变(不同频率点需要调节函数发生器),按照建议的频率点测量CH1$ - $CH2与CH2的相位差,与CH1$ - $CH2、CH2的幅度值,绘制相频曲线与幅频曲线,即$ \varphi-f $图象、$ i-f $图象、$ u-f $图象。
\end{enumerate}

\subsection{观测RLC串联电路的暂态过程}
由函数发生器产生方波,为便于观察,需将方波的低电平调整至与示波器的扫描基线一致。由低电平到高电平相当于充电,由高电平到低电平相当于放电。函数发生器各参数可设置为:频率50\,Hz,电压峰峰值$ u_{pp}=2.0\,\mathrm V $,偏移1\,V。示波器CH1通道用于测量总电压,CH2用来测量电容两端电压$ u_C $,注意两个通道必须共地。实验中$ L=0.1\,\mathrm H,\;C=0.2\,\mu\mathrm F $.
\begin{enumerate}
	\item 当$ R=0\,\Omega $时,测量$ u_C $波形;
	
	\item 调节$ R $测得临界电阻$ R_C $,并与理论值比较;
	
	\item 记录$ R=2\,\mathrm{k}\Omega,\;20\,\mathrm k\Omega $的$ u_C $波形。函数发生器频率可设置为250\,Hz($ R=2\,\mathrm k\Omega $)和20\,Hz($ R=20\,\mathrm k\Omega $)。
\end{enumerate}

\section{实验结果与数据处理}
\subsection{测RLC串联电路的相频特性曲线和幅频特性曲线}
首先改变函数发生器的的输出频率,在示波器中先点击Auto键,在点击“Measure”案件,显示屏左侧选择“相位1-2”(上升沿和下降沿均可),读取相位差的平均值Avg即得CH1和CH2的相位差。当两者相位差最小时,即可测出谐振频率$f_0=2.252\,\mathrm{KHz}$。此时用万用表测得:

\[
    u_L=5.43\,\mathrm{V},\,u_C=\,5.41\,\mathrm{V},\,u=0.468\,\mathrm{V}
\]

由$Q=\frac{u_L}{u}=\frac{u_C}{u}$可以计算$Q$值

\[
    Q_1=\frac{u_C}{u}=11.56,\quad Q_2=\frac{u_C}{u}=11.60
\]

在实验讲义给出的参考频率下,在必要时适当调节函数发生器幅度,在保证路端电压$ u_{pp}=2.0\,\mathrm V $不变的情况下测得电压、电流相位差,以及相应的$ u_R $值(如下页表格所示)。

根据该数据可以作出RLC串联电路的$\phi-f$曲线和$i-f$曲线(见下页)。利用$Q=\frac{f_0}{\Delta F}$算出品质因数。

\newpage

% Table generated by Excel2LaTeX from sheet 'Sheet1'
\begin{table}[htbp]
    \centering
    \begin{tabular}{|c|c|c|c|c|}
    \hline
    $f/\mathrm{KHz}$ & $U(Vpp)/\mathrm{V}$ & $(CH1-CH2)\phi/^{\circ}$ & $u_R(V_{amp})/\mathrm{V}$ & $I_{max}/\mathrm{mA}$ \\
    \hline
    1.88  & 2.00  & -75.64  & 0.351  & 3.51  \\
    \hline
    2.00  & 2.00  & -72.06  & 0.496  & 4.96  \\
    \hline
    2.08  & 2.00  & -61.32  & 0.734  & 7.34  \\
    \hline
    2.15  & 2.00  & -47.88  & 1.01  & 10.1  \\
    \hline
    2.19  & 2.00  & -35.72  & 1.18  & 11.8  \\
    \hline
    2.22  & 2.00  & -18.65  & 1.38  & 13.8  \\
    \hline
    2.24  & 2.00  & -4.028  & 1.42  & 14.2  \\
    \hline
    2.25  & 2.00  & -0.225  & 1.44  & 14.4  \\
    \hline
    2.26  & 2.00  & 5.198  & 1.39  & 13.9  \\
    \hline
    2.275  & 2.00  & 11.73  & 1.34  & 13.4  \\
    \hline
    2.30  & 2.00  & 24.08  & 1.24  & 12.4  \\
    \hline
    2.36  & 2.00  & 46.62  & 0.916  & 9.16  \\
    \hline
    2.43  & 2.00  & 58.20  & 0.728  & 7.28  \\
    \hline
    2.62  & 2.00  & 72.50  & 0.382  & 3.82  \\
    \hline
    3.18  & 2.00  & 74.82  & 0.174  & 1.74  \\
    \hline
    \end{tabular}
    \caption{RLC串联电路实验数据记录表}
\end{table}%

\begin{figure}[htbp]
    \begin{minipage}[t]{0.5\linewidth}
        \centering
        \includegraphics[width=\textwidth]{5-1.png}
        \caption{RLC串联电路的$\phi-f$图}
    \end{minipage}
    \begin{minipage}[t]{0.5\linewidth}
        \centering
        \includegraphics[width=\textwidth]{5-2.png}
        \caption{RLC串联电路的$I_{max}-f$图}
    \end{minipage}
\end{figure}

注:以上两张图像使用Python中的Matplotlib库绘制,用光滑曲线连接散点图只能选取"quadratic"(二次拟合)或"cubic"(三次拟合),故画出的图像不一定精确,线型也不是很符合我的预期。

由$I_{max}-f$图知,$f=2.24\,\mathrm{KHz}$时,$I_{max}$取最大值$14.4\,\mathrm{mA}$。故可设$f_0=2.24\,\mathrm{KHz}$为RLC
串联电路的谐振频率。$I_{max}=\frac{14.4}{\sqrt{2}}\approx10.18\,\mathrm{mA}$时,在Matplotlib的程序运行页面,用鼠标光标可以大致读取对应的频率$f_1=2.152\,\mathrm{KHz}$,$f_2=2.340\,\mathrm{KHz}$,$\Delta f=f_2-f_1=0.188\,\mathrm{KHz}$,$Q=\frac{f_0}{\Delta f}\approx 11.91$。个人认为该结果和上面的$Q_1=\frac{u_C}{u}=11.56,\, Q_2=\frac{u_C}{u}=11.60$结果相近,在误差允许范围内可以接受。可能的误差原因除了读取时的数据跳变意外,还可能有电感、电容的内阻时的实际电路与理想情况并不相同。

\subsection{测RLC并联电路的相频特性和幅频特性曲线}
调节函数发生器输出频率至并联部分电压$ u $与总电流相位相同,即达到谐振,此时可得谐振频率$f_p=2.246\,\mathrm{KHz}$。

在实验讲义给出的参考频率下,在必要时适当调节函数发生器幅度,在保证路端电压峰峰值$ u_{pp}=u+u_{R'}=2.0\,\mathrm V $不变的情况下测得电压、电流相位差,以及相应的$ u_R $值(如下表所示)。根据表格可作出RLC并联电路的$\phi-f$曲线和$u-f$、$i-f$曲线。

注:为求得$u$(CH1-CH2)和$i$的相位差,只能先用示波器光标求得两者峰值之间的时间差,然后利用公式$\phi=\frac{\Delta t}{T}\cdot360^{\circ}=f\Delta t\cdot360^{\circ}$求出相位差。此外原始数据中的$\Delta t$部分与此表相反,因为实验中读取了$i$(CH2)与$u$(CH1-CH2)的时间差,因此需要取反。

% Table generated by Excel2LaTeX from sheet 'Sheet1'
\begin{table}[htbp]
    \centering
    \begin{tabular}{|c|c|c|c|c|c|c|}
    \hline
    $f/\mathrm{KHz}$ & $U(V_{pp})/\mathrm{V}$ & $\Delta t$/μs & $\phi/^{\circ}$ & $u$(Vamp)/V(CH1-CH2) & $u_R$(Vamp)/mV & $I_{max}/\mathrm{mA}$ \\
    \hline
    2.050  & 2.00  & 114  & 84.13  & 1.49 & 894  & 0.1788 \\
    \hline
    2.150  & 2.00  & 104  & 80.50  & 1.69 & 442  & 0.0884 \\
    \hline
    2.200  & 2.00  & 86   & 68.11  & 1.74 & 645  & 0.129 \\
    \hline
    2.231  & 2.00  & 44   & 35.34  & 1.77 & 146  & 0.0292 \\
    \hline
    2.240  & 2.00  & 24   & 19.35  & 1.78 & 117  & 0.0234 \\
    \hline
    2.247  & 2.00  & -2   & -1.618  & 1.79 & 0.724 & 0.0001448 \\
    \hline
    2.250  & 2.00  & -10  & -8.100  & 1.78 & 130  & 0.026 \\
    \hline
    2.253  & 2.00  & -20  & -16.22  & 1.77 & 118  & 0.0236 \\
    \hline
    2.256  & 2.00  & -32  & -25.99  & 1.77 & 120  & 0.024 \\
    \hline
    2.265  & 2.00  & -44  & -35.88  & 1.77 & 158  & 0.0316 \\
    \hline
    2.275  & 2.00  & -58  & -47.50  & 1.76 & 203  & 0.0406 \\
    \hline
    2.320  & 2.00  & -82  & -68.49  & 1.73 & 408  & 0.0816 \\
    \hline
    2.400  & 2.00  & -94  & -81.22  & 1.62 & 778  & 0.1556 \\
    \hline
    2.600  & 2.00  & -92  & -86.11  & 1.18 & 1260 & 0.252 \\
    \hline
    \end{tabular}%
    \caption{RLC并联电路实验数据记录表}
\end{table}%

\begin{figure}[htbp]
    \begin{minipage}[t]{0.5\linewidth}
        \centering
        \includegraphics[width=\textwidth]{5-3.png}
        \caption{RLC并联电路的$\phi-f$图}
    \end{minipage}
    \begin{minipage}[t]{0.5\linewidth}
        \centering
        \includegraphics[width=\textwidth]{5-4.png}
        \caption{RLC并联电路的$u-f$图}
    \end{minipage}
\end{figure}

\newpage

\begin{figure}[htbp]
    \centering
    \includegraphics[width=0.5\textwidth]{5-5.png}
    \caption{RLC并联电路的$I_{max}-f$图}
\end{figure}

其中RLC并联电路的电路图第3个数据点很可能是记录有误,导致画出的图像非常奇怪。

\subsection{观察RLC串联电路的暂态过程}
\begin{itemize}
    \item 调节$R=0\,\Omega$,RLC串联电路处于欠阻尼震荡状态,得到的波形图如下:
    
    \begin{figure}[htbp]
        \centering
        \includegraphics[width=0.9\textwidth]{5-6.png}
        \caption{$R=0\,\Omega$时的波形图}
    \end{figure}

    \item 自小到大调节$R$的大小,当$R=1300\,\Omega$时,波形图的振动基本消失,可以看作看作临界阻尼状态。事实上临界阻尼状态下的电阻值理论上为$R=\sqrt{4L}{C}=\sqrt{4\cdot0.1}{0.2\times10^{-6}}\approx1414.2\,\Omega$,但在实验中,大约在$1200\,\Omega$之后,电阻值每增加$100\,\Omega$,波形的变化非常小。除了波形上难以区分外,造成误差的原因还可能是电容、电感的内阻等。
    \item 按照建议的函数发生器频率调节$R=2\,\mathrm{K}\Omega$和$R=20\,\mathrm{K}\Omega$,电路均处于过阻尼状态。波形图如下所示:
    
    \begin{figure}[htbp]
        \centering
        \includegraphics[width=0.9\textwidth]{5-7.png}
        \caption{$R=2\,\mathrm{K}\Omega$时的波形图}
    \end{figure}

    \begin{figure}[htbp]
        \centering
        \includegraphics[width=0.9\textwidth]{5-6.png}
        \caption{$R=20\,\mathrm{K}\Omega$时的波形图}
    \end{figure}
\end{itemize}


\section{反思总结与心得体会}
\begin{itemize}
    \item 搭建电路是本实验的一大难点,我在上面花费了较多时间。需要注意电学元件的接口问题,电容箱和电阻箱都有三个接线柱,但只有两个接线柱之间是元件,还有一个显示为“接地”的接线柱在本实验中用不到。此外还需要注意示波器CH1和CH2在电流流出电阻处相接,以保证共地。
    \item 在做实验时需要仔细查看要求,虽然讲解过,但是还是很容易做着做着就忘了。此时需要仔细查看讲义上的要求,也需要查看老师讲解时的笔记。我在实验中对于“保持CH1的$u_{pp}$为$2.0\,\mathrm{V}$”理解有误,以为这就是路端电压。事实上函数发生器有$50\,\Omega$内阻,而外部阻抗在谐振点附近时会明显减小,故函数发生器输出电压不等于路端电压。这导致我重新做了一遍实验。
    \item 使用Python的Matplotlib库绘制散点图并用光滑曲线拟合可以得到很漂亮的图,个人认为比Origin和Excel画出的图像更专业也更好看,而且可以直接通过移动鼠标光标直接读取曲线上的点坐标。以后可以多加使用。
    \item 最后,本篇实验报告全部用\LaTeX{}编写,虽然费时费力,但也很大程度上锻炼了我的\LaTeX{}使用水平。这里也感谢21级计算机系的吉骏雄同学和21级人工智能系的林诚皓同学提供了实验报告抬头的模板。
\end{itemize}
\end{document}