\documentclass[12pt]{article}

\usepackage[a4paper]{geometry}
\geometry{left=2.0cm,right=2.0cm,top=2.5cm,bottom=2.5cm}

\usepackage{ctex}
\usepackage{amsmath,amsfonts,graphicx,subfigure,amssymb,bm,amsthm}
\usepackage{algorithm,algorithmicx}
\usepackage[noend]{algpseudocode}
\usepackage{fancyhdr}
\usepackage{mathrsfs}
\usepackage{mathtools}
\usepackage[framemethod=TikZ]{mdframed}
\usepackage{fontspec}
\usepackage{adjustbox}
\usepackage{breqn}
\usepackage{fontsize}
\usepackage{tikz,xcolor}

\setmainfont{Palatino Linotype}
\setCJKmainfont{SimHei}
\setCJKsansfont{Songti}
\setCJKmonofont{SimSun}
\punctstyle{kaiming}

\renewcommand{\emph}[1]{\begin{kaishu}#1\end{kaishu}}

%改这里可以修改实验报告表头的信息
\newcommand{\experiName}{虚拟仪器在物理实验中的应用}
\newcommand{\supervisor}{李贞杰}
\newcommand{\name}{王致力}
\newcommand{\studentNum}{2021K8009908004}
\newcommand{\class}{3}
\newcommand{\group}{03}
\newcommand{\seat}{10}
\newcommand{\dateYear}{2022}
\newcommand{\dateMonth}{10}
\newcommand{\dateDay}{26}
\newcommand{\room}{702}
\newcommand{\others}{$\square$}
%% 如果是调课、补课, 改为: $\square$\hspace{-1em}$\surd$
%% 否则, 请用: $\square$
%%%%%%%%%%%%%%%%%%%%%%%%%%%

\begin{document}

%若需在页眉部分加入内容, 可以在这里输入
% \pagestyle{fancy}
% \lhead{\kaishu 测试}
% \chead{}
% \rhead{}

\begin{center}
    \LARGE \bf 《\, 基\, 础\, 物\, 理\, 实\, 验\, 》\, 实\, 验\, 报\, 告
\end{center}

\begin{center}
    \noindent \emph{实验名称}\underline{\makebox[25em][c]{\experiName}}
    \emph{指导教师}\underline{\makebox[8em][c]{\supervisor}}\\
    \emph{姓名}\underline{\makebox[6em][c]{\name}}%%如果名字比较长, 可以修改box的长度"5em"
    \emph{学号}\underline{\makebox[10em][c]{\studentNum}}
    \emph{分班分组及座号} \underline{\makebox[5em][c]{\class \ -\ \group \ -\ \seat }\emph{号}} (\emph{例}:\, 1\,-\,04\,-\,5\emph{号})\\
    \emph{实验日期} \underline{\makebox[3em][c]{\dateYear}}\emph{年}
    \underline{\makebox[2em][c]{\dateMonth}}\emph{月}
    \underline{\makebox[2em][c]{\dateDay}}\emph{日}
    \emph{实验地点}\underline{{\makebox[4em][c]\room}}
    \emph{调课/补课} \underline{\makebox[3em][c]{\others\ 是}}
    \emph{成绩评定} \underline{\hspace{5em}}
    {\noindent}
    \rule[8pt]{17cm}{0.2em}
\end{center}

\section{实验目的}

\begin{enumerate}
    \item 了解虚拟仪器的概念;
    \item 了解图形化编程语言LabVIEW,学习简单的LabVIEW编程;
    \item 了解伏安法测电阻的虚拟仪器实现方案,并搭建实际电路测量元件的伏安特性曲线。
\end{enumerate}

\section{实验器材}

计算机(Windows 7操作系统),LabVIEW 2014,NI ELVIS II$ ^+ $,导线若干,元件盒一个(包括$ 100\,\Omega $标准电阻一个,待测电阻$ 1\,\mathrm{k\Omega} $和$ 51\,\Omega $各一个,稳压二极管一个)。

\section{实验原理}

\subsection{虚拟仪器的硬件}
本实验使用的硬件平台为个人电脑、美国国家仪器公司(National Instruments)的教学实验室虚拟仪器套件为NI ELVIS II$ ^+ $与其自带的原型板。

\subsection{虚拟仪器的软件}
本实验采用LabVIEW作为用于虚拟仪器系统设计的软件开发平台。LabVIEW将计算机数据分析和显示能力与仪器驱动程序整合在一起,为针对仪器的编程提供了许多便利。同时,与其他常见的计算机高级语言相比,LabVIEW具有图形化的特点,这一点将编程过程简化为设计流程图,实验者即便没有计算机基础,也能在很短时间内的学习中掌握LabVIEW编程的基本操作与初级技巧。

用LabVIEW进行编程的集成开发环境简称VI,其包括三个部分:前面板、程序框图和图标/连线板。

前面板用于设置输入数值与显示输出量,相当于真实仪表的前面板,类比到常见的计算机软件中即为GUI部分(Graphical User Interface,图形用户界面)。前面板上的图标分为输入类(Controls)和显示类(Indicators),具体而言可以是开关、旋钮、图形、图表等表现形式。

程序框图则相当于仪器的内部功能结构,类似常见计算机软件中的后台进程。程序框图中的端口用于和前面板的输入对象和显示对象传递数据,节点用于实现函数与功能子程序调用,图框用于实现结构化程序控制命令,连线则代表程序执行过程中的数据流。

\subsection{利用虚拟仪器测量伏安特性}
本实验采用伏阻法测量电路的伏安特性曲线,利用可变压电源、两个电压表和一个定值电阻进行测量。测量电路如下图所示:

\begin{figure}[htbp]
    \centering
    \includegraphics[width=0.6\textwidth]{0-1.png}
    \caption{用虚拟仪器测量伏安特性原理图}
\end{figure}

设路端电压为$U_0$,定值电阻电阻为$R_0$,两端电压为$U_1$。由串联电路的分压原理可以得到待测电阻上的电压,对标准电阻根据欧姆定律可以得到串联电路中的电流$ I $。由欧姆定律即,可求得待测电阻的阻值:

\[R=\frac{U_0-U_1}{U_1}R_0\]

具体连线方式将在实验内容部分讨论。

\section{实验内容}

\subsection{初步熟悉LabVIEW开发环境的基本操作和编程方法}
启动LabVIEW2014之后,在左上角“文件”下拉菜单中点击“新建VI”即可得到新项目的前面板。在上方菜单栏点击“窗口-显示程序框图”即可得到新项目的程序框图,可以用分屏功能将两者放在屏幕左半边和右半边。在前面板点击右键会弹出控件选板,在程序框图点击右键会弹出函数选板。在这两个选板界面点击相应控件或者函数,可以把相应控件或函数拖拽到窗口的相应位置,而插入控件后程序框图中也会显示这一控件的图标。经过一定自由探索,不难掌握标签工具、定位工具、连线工具、快捷键等其他功能。

\subsection{创建一个模拟温度测量程序}
该温度测量程序可以实现“摄氏”和“华氏”两种模式,同时在前面版中用温度计图标进行图形化显示。

一种方案不利用实际的温度传感器,温度由使用者自行设定,具体实验步骤如下:

\begin{enumerate}
    \item 新建VI项目,在前面板中,放入温度计、垂直滑动杆开关、数值显示控件、数值输入控件,并利用标签工具修改其名称以表示所需的含义。
    \item 在程序框图中,在函数选板中找到乘法、减法、除法、比较函数,将其与相应的图标通过连线工具连接起来,并在需要的地方创建数值常量。整个程序创建完毕后,最后整理图标位置与连线,使得程序框图更为美观。
    \item 运行程序:选择前面板窗口,点击连续运行按钮,改变“采集的电压”控件输入值和温度值单位,观察程序运行情况。再次点击连续运行按钮即可停止程序运行。
    \item 保存该VI项目,关闭程序。
\end{enumerate}

\begin{figure}[h!]
    \centering
    \includegraphics[width=0.92\textwidth]{2-0.png}
    \caption{不使用温度传感器的模拟温度测量程序}
\end{figure}

在程序运行时,“采集的电压”在$0.5-2$之间取值。当开关接到“华氏度”模式时,测到的电压乘以$100$即得温度计示数;当开关接到“摄氏度”时,测到的电压先乘以$100$转换为华氏度,然后先减$32$再除以$1.8
$得到摄氏度。

另一种测量装置使用了热电偶作为实际的温度传感器。在电路中我们将它接在开发板的"AI0"正负两端,在程序中我们使用DAQ助手取代上图中的“采集的电压$\times100$”,具体实现方法参见讲义。前面板和程序框图如图所示:

\begin{figure}[h!]
    \centering
    \includegraphics[width=0.92\textwidth]{2-4.png}
    \caption{使用热电偶作为温度传感器的模拟温度测量程序}
\end{figure}

注意到热电偶的读数不稳定,跳动很大,而且开发板AI0接口和热电偶引脚接触不紧密时读书容易变得非常大,我们需要手动将引脚接紧,方便得到相对正常的读数。

\subsection{创建一个电压输出和采集的程序}
我们用开发板的AO0端口将程序设定的电压输出,再将上面输出的电压用导线接到开发板的AI0端读取,并用程序处理和输出。前面板和程序框图如图所示:

\begin{figure}[h!]
    \centering
    \includegraphics[width=0.7\textwidth]{3-1-1.png}
\end{figure}

\begin{figure}[h!]
    \centering
    \includegraphics[width=0.7\textwidth]{3-1-2.png}
    \caption{电压输出和采集程序}
\end{figure}

实验步骤如下:

\begin{enumerate}
    \item 新建VI项目。
    \item 编写电压输出程序。在程序框图中创建虚拟通道,将其类型设置为模拟输出电压。将“DAQmx开始任务”“DAQmx写入”和“DAQmx清除任务”三个节点放入程序框图并将其连接,在“DAQmx写入”的图标数据输入端创建输入控件。创建一个While循环结构,使得每等待100\,ms执行一次循环结构内部分,直至在前面板按下停止按钮。
    \item 编写电压采集程序。在程序框图中创建虚拟通道,将其类型设置为模拟输入电压。以类似的方式构建任务链与While循环,不同之处在于此处不采用“DAQmx写入”控件而是“DAQmx读取”控件,其连接输出控件而非输入控件。
    \item 通过USB线连接PC与原型板,打开ELVIS电源与原型板电源,在前面板上设置输出通道为Dev5/ao0(本套原型板在电脑上显示为Dev5),输入通道为Dev5/ai0。在原型板上用原型板连接模拟输出的“AO 0”端和模拟输入的“AI $ 0+ $”端,将“AI $ 0- $”端与接地端“AIGND”用导线连接。在前面板窗口运行VI程序,不断改变输出电压,观察测量电压是否与输出电压一致,也可点击停止按钮观察程序运行情况。
    \item 停止程序运行,保存VI项目,关闭程序。
\end{enumerate}

在程序运行过程中,如果不点击“停止输出”或“停止测量”,则任意设定输出电压值,在和设定完成几乎同时的情况下,测量电压发生改变且恒等于输出电压。如果点击“停止输出”或“停止测量”,此时无论输出电压如何改变,测量电压都维持在结束前的值不改变。此时需要把“停止输出”和“停止测量”都点一遍,输出值和读取值才能做到同步。

\subsection{用虚拟仪器测量伏安特性}
这是本次实验中最复杂的一个。实验步骤如下:
\begin{enumerate}
    \item 创建前面板。在前面板上放上Express XY图用于显示电压-电流图。放入四个数值型输入控件,分别用于输入“输出电压步长”、“测量数据点数”、“标准电阻”、“时间间隔”四个参数,并规定时间间隔与输入电压步长的单位。放入一个数值型显示控件用于显示电阻测量值,放入一个二维数组显示控件用于显示测量的电压和电流。最后放入一个开关用于控制程序进程。
    \item 创建程序框图。放入一个5帧的顺序结构,第0帧用于让ELVIS输出电压,第1帧让程序等待100\,ms至电路稳定,第2帧用于让ELVIS采集总电压与标准电阻上的电压,第3帧再次让程序等待100\,ms以减少对数据测量过程的影响,第4帧令顺序结构结束时电压输出为0。这样的顺序结构实现了测量一次的过程,为了不断改变输出电压值来测量电阻的值,还需加入一个While循环结构,使得While循环及其中的顺序结构在执行次数i不大于测量数据点数时循环执行。为了实现数据的实时显示,我们使用两个移位寄存器,以在相邻两次循环之间传递数据。通过一个“创建数组”来显示测量数据,将电压、电流数组分别与“创建XY图”的X、Y输入相连。为求得电阻值,需在循环外放入一个“线性拟合”节点,将移位寄存器传递出来的电流、电压数组与“线性拟合”的X、Y输入端相连,将其“斜率”输出端与“待测电阻值”显示控件相连,用于显示电阻值。
    \item 正确连接外部电路,根据实验原理部分中给出的原理电路图与上一步DAQ助手设置的输入输出端口,在原型板上连接实验电路。
    \item 在前面板上设置各参数,在电路中依次插入两个电阻和一个二极管,将测得数据导出为excel文件,绘制三种元件的伏安特性曲线。
\end{enumerate}

\begin{figure}[htbp]
    \centering
    \includegraphics[width=0.92\textwidth]{4-3.png}
    \caption{测量伏安特性曲线程序的前面板部分}
\end{figure}

\begin{figure}[htbp]
    \centering
    \includegraphics[width=0.92\textwidth]{4-4.png}
    \caption{测量伏安特性曲线程序的程序框图部分}
\end{figure}

在连接电路时,路端电压由AO1提供,即电流从AO1流出,最后流入GROUND。电压的测量需要用到两个模拟输入端口:AI0的正负极接在电路正负极之间,测量路端电压;AI1的正负极接在定值电阻两端,测量定值电阻的电压。

\begin{figure}[htbp]
    \centering
    \includegraphics[width=0.7\textwidth]{4-5-1.jpg}
    \caption{测量伏安特性曲线实验所用电路示意图}
\end{figure}

\begin{figure}[htbp]
    \centering
    \includegraphics[width=0.7\textwidth]{4-5-3.jpg}
    \caption{测量伏安特性曲线实验所用实际电路图(以接入电阻为例,接入二极管同理)}
\end{figure}

%\pagebreak

\section{数据及数据处理}

\subsection{电阻的伏安特性曲线}
实验中要求使用$1\,\mathrm{k}\Omega$和$51\,\Omega$的电阻,然而我所在的桌上并没有$1\,\mathrm{k}\Omega$的电阻,故我只能用其他电阻代替。用万用表大致测量可得,电阻1的阻值约为$51.2\,\Omega$,电阻2的阻值约为$1492\,\Omega$。

\subsubsection{电阻1的伏安特性曲线}
将电阻1接入电路,得到如下表格:

\begin{table}[htbp]
    \centering
    \begin{tabular}{|c|c|}
    \hline
    电压(V) & 电流(A) \\
        \hline
        -0.000644323 & -1.61E-05 \\
        \hline
        0.0328605 & 0.000644284 \\
        \hline
        0.0660431 & 0.00130149 \\
        \hline
        0.0995479 & 0.00196192 \\
        \hline
        0.133697 & 0.00261591 \\
        \hline
        0.16688 & 0.00327312 \\
        \hline
        0.200385 & 0.00392711 \\
        \hline
        0.234856 & 0.0045811 \\
        \hline
        0.267394 & 0.00524797 \\
        \hline
        0.300899 & 0.00590196 \\
        \hline
        0.334082 & 0.00655917 \\
        \hline
        0.367909 & 0.00721638 \\
        \hline
        0.401736 & 0.00787037 \\
        \hline
        0.434275 & 0.0085308 \\
        \hline
        0.468102 & 0.00918801 \\
        \hline
        0.501607 & 0.009842 \\
        \hline
    \end{tabular}%
    \caption{电阻1的伏安特性数据记录表}
\end{table}%

用Excel绘制散点图,将数据列添加趋势线,对散点进行线性拟合,得到电阻1的$I-U$图线(伏安特性曲线):

\begin{figure}[h!]
    \centering
    \includegraphics[width=0.75\textwidth]{5-1.png}
    \caption{电阻1的伏安特性曲线}
\end{figure}

根据线性元件的欧姆定律,$I-U$图线的斜率$k=\frac{1}{R}=0.0196\,\Omega^{-1}$,故$R=51.02\,\Omega$。而虚拟实验面板的读数为$50.96\,\Omega$,两者在误差允许范围内相等。

\subsubsection{电阻1的伏安特性曲线}
将电阻2接入电路,得到如下表格:

\begin{table}[htbp]
    \centering
    \begin{tabular}{|c|c|}
        \hline
        电压(V) & 电流(A) \\
        \hline
        -0.00128865 & -6.48E-06 \\
        \hline
        0.0927825 & 5.47E-05 \\
        \hline
        0.186209 & 0.0001192 \\
        \hline
        0.279636 & 0.0001836 \\
        \hline
        0.372741 & 0.000248 \\
        \hline
        0.466812 & 0.0003125 \\
        \hline
        0.559917 & 0.0003769 \\
        \hline
        0.653666 & 0.0004349 \\
        \hline
        0.748381 & 0.0004961 \\
        \hline
        0.840842 & 0.0005637 \\
        \hline
        0.935235 & 0.000625 \\
        \hline
        1.02834 & 0.0006862 \\
        \hline
        1.12306 & 0.0007474 \\
        \hline
        1.21584 & 0.0008086 \\
        \hline
        1.30991 & 0.000873 \\
        \hline
        1.40334 & 0.0009342 \\
        \hline
    \end{tabular}%
    \caption{电阻2的伏安特性数据记录表}
  \end{table}%

用Excel绘制散点图,将数据列添加趋势线,对散点进行线性拟合,得到电阻2的$I-U$图线(伏安特性曲线):

\begin{figure}[h!]
    \centering
    \includegraphics[width=0.75\textwidth]{5-2.png}
    \caption{电阻2的伏安特性曲线}
\end{figure}

根据线性元件的欧姆定律,$I-U$图线的斜率$k=\frac{1}{R}=0.0007\,\Omega^{-1}$。由于有效数字位数过低,我们直接利用虚拟实验仪器面板的读数$R=1492.66\,\Omega$。由于$k\cdot R\sim 1.0$,故虚拟仪器的读数可信。

\subsubsection{二极管的伏安特性曲线}
将稳压二极管接入电路,得到如下表格:

用Excel绘制散点图,将数据列添加趋势线,对散点进行线性拟合,得到二极管的$I-U$图线(伏安特性曲线):

\begin{table}[htbp]
    \centering
    \begin{tabular}{|c|c|}
        \hline
        电压(V) & 电流(A) \\
        \hline
        -1.49934 & -6.48E-06 \\
        \hline
        -1.39883 & -6.48E-06 \\
        \hline
        -1.29831 & -6.48E-06 \\
        \hline
        -1.19844 & -6.48E-06 \\
        \hline
        -1.09889 & -6.48E-06 \\
        \hline
        -0.998057 & -3.26E-06 \\
        \hline
        -0.899475 & -9.70E-06 \\
        \hline
        -0.799927 & -9.70E-06 \\
        \hline
        -0.699091 & -9.70E-06 \\
        \hline
        -0.597932 & -3.26E-06 \\
        \hline
        -0.499028 & -6.48E-06 \\
        \hline
        -0.39948 & -1.29E-05 \\
        \hline
        -0.298966 & -6.48E-06 \\
        \hline
        -0.198451 & -3.26E-06 \\
        \hline
        -0.0995479 & -9.70E-06 \\
        \hline
    \end{tabular}%
    \qquad
    \begin{tabular}{|c|c|}
        \hline
        电压(V) & 电流(A) \\
        \hline
        0.0989035 & -6.48E-06 \\
        \hline
        0.199096 & -6.48E-06 \\
        \hline
        0.297999 & -3.88E-08 \\
        \hline
        0.397225 & 9.63E-06 \\
        \hline
        0.488397 & 9.98E-05 \\
        \hline
        0.555406 & 0.0004155 \\
        \hline
        0.597288 & 0.0009987 \\
        \hline
        0.624994 & 0.0017139 \\
        \hline
        0.644646 & 0.0025096 \\
        \hline
        0.658499 & 0.0033633 \\
        \hline
        0.668486 & 0.0042396 \\
        \hline
        0.679117 & 0.005132 \\
        \hline
        0.688138 & 0.0060341 \\
        \hline
        0.69426 & 0.0069587 \\
        \hline
        0.699737 & 0.0078833 \\
        \hline
    \end{tabular}%
    \caption{二极管的伏安特性数据记录表。左侧为加反向电压时的情况,右侧为加正向电压时的情况。}
\end{table}%

用Excel绘制散点图并用光滑曲线连接,得到二极管的$I-U$图线(伏安特性曲线):

\begin{figure}[h!]
    \centering
    \includegraphics[width=0.75\textwidth]{5-3.png}
    \caption{二极管的伏安特性曲线}
\end{figure}

该图像正确地反映了二极管的理论伏安特性曲线特点:具有单向导通性,加反向电压时电阻非常大;加正向电压时二极管大约在$0.5V$时导通,此后电阻迅速减小。

\section{思考题}
\begin{enumerate}
    \item 虚拟仪器系统与传统仪器有什么区别?请简要说明。
    
    {\kaishu 虚拟仪器在以 PC 为核心组成的硬件平台支持下,通过软件编程来实现仪器的功能,虽然一般也需要搭建实际电路,但电压输出和读取、简单数据处理等功能都是可以用软件进行操作的。而传统仪器为实物仪器,测量和处理数据需手动操作。传统仪器在实验时常常会因读数不便、调整不变、自动化程度低、过于依赖人工等因素给实验者带来许多苦恼与压力,而虚拟仪器和LabVIEW软件的灵活高效解决了这些问题,同时提供了高精度、高效率等优点。}
    \item 本实验内容 3 中的电压输出和采集哪个先执行?
    
    {\kaishu 先执行电压输出,在输出后还需等待 100ms 再进行电压采集,这是为了使信号稳定。}
\end{enumerate}

\section{实验感想}
本实验使用的 LabVIEW 由前面板和程序框图组成,采用图形化编程的思想,和小学时使用的Scratch软件具有一定的相似性,直观而且简便。作为一名计算机系的学生,我虽然对于LabVIEW的各种功能了解非常有限,不过还是可以从这次实验中感受到这一技术的趣味性,也对于实验所用的原型板、图形化编程背后的接口等产生了一定的兴趣。

虚拟仪器使得本实验的数据采集和实验非常简便,数组中的数据甚至可以直接导出为Excel,省去了很多数据输入和数据处理的工作,且自带线性拟合功能,可以直接从界面上读取电阻的数值。同时收集的数据也非常精确,测量的电压和电流值有效数字位数非常多,在两个电阻伏安特性曲线的线性拟合中,拟合结果的$R^2$都至少为$0.9999$。这和传统实验相比的优势是非常明显的。比如上次的磁滞回线实验需要测量的数据点估计有50个左右,都需要用光标手动读取电压值。但电压只是一个中间变量,我们需要将其转换为$H$和$B$,然而示波器却没有对于数据进行自由四则运算的功能,导致看到图像后不能像本次实验一样的到磁滞回线图像,还需要测量图像上的点的坐标,之后所有电压测量值全部需要手动转换为$H$和$B$,最后还需要重新绘制图像,非常不方便,又费时又费力。虚拟仪器的优势可见一斑。

但虚拟仪器也有自身的缺陷,比如编程复杂,不仅需要搭建电路,还需要实现复杂的程序控制功能,尤其是数据采集、存储、回归分析和输出的程序非常复杂。相比之下传统的伏安特性曲线测量装置虽然功能简陋、不能自动控制,但是搭建过程非常简单,大约两三分钟就可以。此外实验装置的成本也显著变高了,现在的单片机(比如Arduino,STM32,Raspberry Pi)都很贵,网上没有查到NI ELVIS II$ ^+ $套件的具体价格,但由于这个比上述单片机都更加专业,显然价格会更高。

不过这次实验完全不顺利,尤其是在进行第四个实验时,由于讲义上很多细节没有写清楚(比如怎么加移位寄存器,怎么给寄存器连线),我又一次最后一个走,而且做到了6点,比上次磁滞回线都晚。不过还是很感谢指导老师等了我很久,帮我重新搭建了电路,修好了实验4中遇到的bug。

此外需要注意的是,由于电路采用串联的方式,故不可以在实验4中使用太大的电阻,尽量按照$1\,\mathrm{k}\Omega$和$51\,\Omega$的标准来,否则电压几乎全部集中在待测电阻上,测出的结果不准。

最后,本篇实验报告全部用\LaTeX{}编写,虽然费时费力,但也很大程度上锻炼了我的\LaTeX{}使用水平。这里也感谢21级计算机系的吉骏雄同学和21级人工智能系的林诚皓同学提供了实验报告抬头的模板。
\end{document}